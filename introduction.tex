\chapter{Introduction}

%\begin{figure}[htbp!]
%    \centering
%
%    \begin{tikzpicture}[scale=2.5]
%        \squareworstcase
%    \end{tikzpicture}
%
%    \caption{Worst-case instance for packing density?}
%    \label{fig:worst-case}
%\end{figure}
%
%Consider the circle packing shown in \Cref{fig:worst-case}. For these two circles, the surrounding square is the smallest one in which the circles can be packed. Originally, we considered the following problem: Does this instance constitute the worst case in terms of packing density, or is there a set of circles with the same combined area that cannot be packed into this square? Put differently, can all other sets of circles with the same combined area as these two
%(like the one in \Cref{fig:big-question})
%be packed into this specific square?
%We will give a constructive proof that this is indeed the case.
%
%\begin{figure}[htbp!]
%    \centering
%
%    \begin{tikzpicture}[scale=2.5]
%        \bigquestion
%    \end{tikzpicture}
%
%    \caption{Can these circles be packed?}
%    \label{fig:big-question}
%\end{figure}

%Furthermore, we will describe an algorithm which is able to pack circles into non-acute triangles with worst-case density.

\section{Results}

We give a constructive proof that all circle instances with a combined area of $a$ can be packed into all non-acute triangles with an incircle not larger than $a$.

Furthermore, ...

\section{Related work}

Packing objects into a container is a classic, intuitive problem...

Most works concerned with circle packing follow one of the two following approaches:

Either they model the packing problem as an optimization problem, and solve it using commercial nonlinear solvers.

Or they apply heuristics.

In this thesis, we are interested in approximation algorithms, which give certain performance guarantees on the obtained results.

\paragraph{History}

The packing of equal circles in a square has received particular attention in the past.

The optimal packings of up to nine equal circles were already proven in 1965.
\textcite{schaer1965densest} proofed the optimal solutions for $n = 7$ and $n = 8$, and \textcite{SM1965geometric} the one for $n = 9$.
The optimal solution for $n = 10$ was not confirmed until 1990 by \textcite{DPW1990optimal}. In the same paper, optimal solutions for 11, 12 and 13 circles were also proven.
See \textcite{WMP1994history} for an overview of the history of optimal solutions for $n \le 13$.

The best known solutions for packing equal circles into squares, circles, rectangles, and other containers are continuously published on \url{packomania.com} \cite{specht2015packomania}.

\paragraph{Equal circles in rectangles}

\paragraph{Unequal circles in squares}

The problem of finding the smallest square that suffices for packing any set of circles of total area 1 was posed by \textcite{DFL2010circle}. They described a quadtree-based approximation scheme which could guarantee a packing density of $\pi/16 \approx 0.196349$.


\paragraph{Equal circles in circle}

\textcite{lubachevsky1997curved} studied \emph{curved hexagonal packings} of equal circles in a circle, a pattern derived from known best packings of circles in hexagons. For large $n$, a curved hexagonal packing was found to be not optimal. Instead, good packings found by the authors seemed to have a curved hexagonal-packed core and an irregular pattern along the periphery.

\paragraph{Unequal circles in circle}

\textcite{WHZX2002improved} describe a “quasi-physical, quasi-human” approach, which first simulates the circles according to gravity and then finds circles which are "most squeezed", and re-insert them randomly.

\paragraph{Circles in strips}

\paragraph{Hardness}

The decision problem whether it is possible to pack a given set of circles into a square or an equilateral triangle was shown to be NP-complete by \textcite{DFL2010circle}.

\subsection{Packing squares}

Approximation algorithms which pack squares are interesting in this context as we can directly use them for packing circles: We simply embed each circle with radius $r$ in a square of edge length $2r$.

This means that if we are given an algorithm which can pack all square instances with a total area of $a$ into a given shape, we immediately get an algorithm which can pack all circle instances with a total area of $\frac{\pi}{4}a$.

\paragraph{Packing squares into a rectangle}

The decision problem whether it is possible to pack a given set of squares into a square container was shown to be strongly NP-complete by \textcite{LTWYC1990packing} through a reduction from \textsc{3-PARTITION}.

\textcite{MM1967some} developed an algorithm which can pack all square instances with a total area of $1/2$ into the unit square.
%proved a critical density of $\nicefrac{1}{2}$ when packing squares into a square.
They sort the squares by their size and place them into the square in a shelf-like manner, see fig TODO.
They also proofed that $1/2$ is the critical density for packing squares into a square: Two squares, each with an area of more than $1/4$ of the container's area, cannot be packed, see fig TODO.
As by above remark, their algorithm can also be used to pack circles: It can pack all circle instances with a total area of $\frac{\pi}{8} \approx 0.39269908$ into the unit square.

\textcite{KK1975optimal} showed that any square instance of total area 1 can be packed into a rectangle of dimensions $2/\sqrt{3}$ times $\sqrt{2}$, and that this is the smallest-area rectangle with that property. The packing density in this case is 0.612372435695. Applied to circles, this result means that all circle instances with a total area of 1 can be packed into a rectangle of dimensions $4/\sqrt{3\pi}$ times $2\sqrt{2}/\sqrt{\pi}$.

\textcite{hougardy2011packing} showed that for any set of squares with a total area of 1, there is some rectangle with an area $< 1.4$ in which the squares can be packed using a computer generated proof.

%Online
%Upper bound:
%Lower bound:

\paragraph{Heuristics}

\textcite{LIE2014approximate} compute approximated solutions to packing circles into rectangles by restricting their coordinates to a regular grid and formulate a binary LP, where the variables represent the assignment of the centers to the grid nodes.

\textcite{ZYC2015packing} extend the GVS to allow packing into \emph{damaged squares}, squares with some smaller squares removed from it. They enhance the GVS by Simulated Annealing and get significantly better results in their experiments compared to GVS.

\paragraph{Unsorted}

\textcite{MPSSW2014polynomial} devised a general polynomial-time approximation schemes for packing circle-like objects into containers, which first packs a constant number of large objects, and then fills up the unused space with the small objects.

\section{Applications}

Circle packing has all kinds of useful applications in engineering, science and everyday life.

% categories: coverage, storage, packaging, tree exploitation, cutting industry

Running optical fibers through a tube or placing many differently-sized pipes through a smaller one \parencite{WHZX2002improved}

For example, car manufacturers may want to drill circular holes into a car's body to route a number of sensor cables through that hole, but keep the hole as small as possible not to weaken the car's body \cite{SSSKK2004disk}.

Placing radio towers for maximal coverage \parencite{SMCSCG2007new}.

Cutting circular pieces from a material, minimizing waste, loading a container (like a vehicle or a parcel with cylindrical objects, minimizing the number of nodes in a geographical communication network or dispersing facilities so that the minimum distance between any two is maximized, the layout of control panels containing circular controls (like in airplanes) \parencite{CKP2008solving}

Other applications stem from chemistry, where one can study dense packing of atoms in crystals or macromolecules \cite{WMP1994history}.
% or info processing (sperical codes)
%https://en.wikipedia.org/wiki/Quadrature_amplitude_modulation

Photo collage "Content-aware photo
collage using circle packing"

Cutting a given set of disks from a piece of material, while wasting as little as possible of it \cite{SMCSCG2007new}

Foresting, where one wants to plant trees so that the forest will be as dense as possible, but the trees do not hinder each other's growth \cite{SMCSCG2007new}.

%storage of a cylindrical drums into containers or
%stocking them into an open area, packaging bottles or cans into the smallest box,

%"Other applications one can find in the motor
%cycle industry, communication networks, facility location and
%dashboard layout"

Another application includes the design of crease patterns for Origami, where one wants to find a sequence of folds of a square piece of paper, so that the result resembles a specific shape. This kind of Origami design was pioneered by citeTODO. Using circle packings, one can find crease patterns for arbitrary stars.
