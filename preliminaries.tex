\chapter{Preliminaries}

\section{Circle instances}

We start off with some definitions which make our life easier when talking about circle instances:

\begin{definition}
    For each $0 \le a$, an \emph{$a$-circle} is a circle with an area of $a$.
\end{definition}

\begin{definition}
    A \emph{circle instance} is a multiset of non-negative real numbers, which define the circles' areas. Addition of circle instances is defined like on multisets.
    For any circle instance $C$, $\mysum(C)$ is the combined area of the instance's circles and $\min(C)$ is the area of the smallest circle contained in the instance.
\end{definition}

For example, for the circle instance $C = \{3,2,1,4\}$, $\mysum(C) = 10$ and $\min(C) = 1$.

\begin{definition}
    $\C$ is the set of all circle instances. $\C(a)$ consist of exactly those circle instances $C$ with $\mysum(C) \le a$. Finally, $\C(a,b)$ consists of exactly those circle instances $C \in \C(a)$ with $\min(C) \ge b$.
\end{definition}

\section{Shapes}

\begin{definition}
    A shape's \emph{incircle} is the largest circle that can be packed into the shape.
\end{definition}

\begin{definition}
    A shape's \emph{twincircles} are the maximum-area circle instance consisting of exactly two equal circles that can be packed into the shape. The shape's \emph{twincircle area} is their combined area.
\end{definition}
