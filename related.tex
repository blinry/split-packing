\chapter{Related work}

Packing problems have received a lot of attention in the past. The related work can roughly be categorized according to their objective when approaching a packing problems: Some authors strive for optimal solutions, some apply heuristics and optimization methods in order to attain “good” solutions, and some design approximation algorithms, which make performance guarantees about the obtained solutions, in the best case resulting in a constant approximation factor.

A packing problem usually consists of three components: The type of objects that are to be packed (for example, squares or circles, which can be different-sized or being restricted to be of equal size), the type of the container (for example, a square, a triangle, a general polygon), and the restrictions which are placed on the movement of the objects (they could be allowed to rotate, or restricted to translations only). In this literature review, as well as in the rest of this thesis, we always restrict the movement of the objects to translations.

Packing arbitraryly-shaped objects into arbitrary containers can immediately shown to be NP-hard: Consider a rectangular container of dimensions $2 \times n$ and rectangular objects of height $1$, whose widths add up to $n$. The decision problem whether it is possible to pack these objects into the container without rotation is a reduction from \textsc{Partition}.

\section{Packing squares}

We start by briefly reviewing related work which deals with packing squares into various containers.
Algorithms which pack squares are interesting in the context of this thesis as we can directly use them for packing circles: We embed each circle with radius $r$ in a square of edge length $2r$ and pack the resulting squares to receive a packing of the original circle instance.

\subsection{Hardness}

Packing problems remain hard for equally-sized squares and arbitrary containers: \textcite{FPT1981optimal} showed it to be NP-hard to pack equal squares into arbitrary polygons with holes by a reduction from \textsc{3-SAT}. They construct gadgets for variables and clauses and connected them with wires. In Fig TODO, such a construction is depicted.

One can get a seemingly “simple” packing problem by restricting the container as well as the objects to (not necessarily equal) squares.
The decision problem whether it is possible to pack a given set of squares into a square container was shown to be strongly NP-complete by \textcite{LTWYC1990packing} through a reduction from \textsc{3-Partition}.

\subsection{Approximation algorithms}

If we are given an algorithm which can pack all square instances with a total area of $a$ into a given shape, we immediately get an algorithm which can pack all circle instances with a total area of $\frac{\pi}{4}a \approx 0.7853981a$ into that shape. Thus, square packing approximation algorithms provide us with lower density bounds for optimal circle packings.

\textcite{MM1967some} developed an algorithm which can pack all square instances with a total area of $1/2$ into the unit square.
They sort the squares by their size and place them into the square in a shelf-like manner, see fig TODO.
They also proofed that $1/2$ is the critical density when packing squares into a square: Two squares, each with an area of more than $1/4$ of the container's area, cannot be packed, see fig TODO.
As by above remark, this algorithm can be used to pack circles: It can pack all circle instances with a total area of $\pi/8 \approx 0.39269908$ into the unit square.

\textcite{KK1975optimal} showed that any square instance of total area 1 can be packed into a rectangle of dimensions $2/\sqrt{3} \times \sqrt{2} \approx 1.15470053 \times 1.4142135$, and that this is the smallest-area rectangle with that property. The packing density in this case is 0.6123724. Applied to circles, this result means that all circle instances with a total area of 1 can be packed into a rectangle of dimensions $4/\sqrt{3\pi} \times 2\sqrt{2}/\sqrt{\pi}$.

\textcite{hougardy2011packing} showed that for any set of squares with a total area of 1, there is some rectangle with an area $< 1.4$ in which the squares can be packed using a computer generated proof. This means that for any set of circles with a total area of 1, there is some rectangle with an area $< 28\pi/5 \approx 1.7825353$ in which the circles can be packed.

\section{Packing circles}

Most works concerned with circle packings try to get as close to the optimal packings of given instances as possible. Optimal solutions are only known for small instances, so that the majority of authors either focus on heuristics or model circle packing as a global optimization problem.

%In this thesis, on the other hand, we are interested in approximation algorithms, which give certain performance guarantees on the obtained results.

\subsection{Hardness}

The problem we are concerned with in this thesis is packing (unequal) circles into simple containers, like squares and triangles. The decision problem whether it is possible to pack a given set of unequal circles into a square or an equilateral triangle was shown to be NP-complete by \textcite{DFL2010circle}, also by a reduction from \textsc{3-Partition}. They build a circle instance which first forces some symmetrical free “pockets” in the resulting circle packing. The instance's remaining circles can be packed into these pockets exactly if the related \textsc{3-Partition} problem can be solved.

\subsection{Density bounds}

The densest packing of equal circles in the plane can be easily guessed: Arrange them on a hexagonal grid, so that each circle touches six others. This arrangement results in a packing density of $\pi/\sqrt{12} \approx 0.90689968$. An incomplete proof was given by \textcite{thue1892om}, which was later fixed by \textcite{fejestoth1940uber}. For a simple proof, see \cite{CW2010simple}.

For unequal circles, the optimal packing density can get arbitrarily close to 1 if the circle instance is chosen so that it can be packed in the manner of an Apollonian gasket, or in another space filling manner, like the construction of \textcite{bourke2011random}.

\subsection{Optimal packings}

The packing of equal circles in a square has received particular attention in the past.

A problem which has been considered a lot is is what is the smallest square in which $n$ unit circles can be packed. Surprisingly, optimal results are only known for $n < 35$.
%TODO: correct?
The optimal packings of up to nine equal circles in a square were already proven in 1965: The cases for $n < 7$ are trivial TODOcite,
\textcite{schaer1965densest} proofed the optimal solutions for $n = 7$ and $n = 8$, and \textcite{SM1965geometric} the one for $n = 9$.
The optimal solution for $n = 10$ was not confirmed until 1990 by \textcite{DPW1990optimal}. In the same paper, optimal solutions for 11, 12 and 13 circles were also proven.
See \textcite{WMP1994history} for an overview of the history of optimal solutions for $n \le 13$.
The smallest number of equal circles for whose optimal packing in a square is currently not known is TODO.
\textcite{NO1998more} provide computer-aided optimality proofs for $n \le 27$. Their algorithm, which subdivides the square into square or rectangular tiles, is based on the procedure by \textcite{PWMD1992packing}.
\textcite{MC2005new} give computer-aided proofs for $n=28,29,30$ within very tight tolerance values.
Furthermore, \textcite{LR2002packing} proved the known solutions for up to $n=35$ to be optimal within very small tolerance values.

\subsection{Approximation algorithms}

The problem of finding the smallest square that suffices for packing any set of circles of total area 1 was posed by \textcite{DFL2010circle}. They described a quadtree-based approximation scheme which could guarantee a packing density of $\pi/16 \approx 0.196349$.

\textcite{MPSSW2014polynomial} devised a general polynomial-time approximation schemes for packing circle-like objects into containers, which first packs a constant number of large objects, and then fills up the unused space with the small objects.

\textcite{HMS2016bounded} devised an asymptotic approximation algorithm, which can pack unequal circles into square bins in an online fashion with an asymptotic competitive ratio of at most 2.4394. They also gave a lower bound of 2.2920 for any online bounded space algorithm for that problem. The algorithm packs large circles according to the best known packings of equal circles and puts small circles into recursively subdivided hexagonal bins.

\subsection{Heuristics}

If $n$ is too large to determine optimal packings, one can ask for lower bounds for the density of optimal packings, that is, the smallest \emph{known} containers in which the circles can be packed.
Many authors employ heuristics to produce good solutions for circle packing problems.
A considerable amout of work has been put into finding new records, mostly for equal circles, but also for test instances consisting of unequal circles. The best known solutions for packing equal circles into squares, circles, rectangles, and other containers are continuously published on \url{packomania.com} \cite{specht2015packomania}.

The works related to this idea are too numerous to be covered here in their entierty, but we will present some of the more important attempts in roughly chronological order. As described, these approaches do not give any performance guarantees, but give lower bounds on the optimal packing of single instances. Most of the works lead to improvements of the best known packings in the literature.

\paragraph{Quasi-human or quasi-physical}

As one of the first to consider the packing of unequal circles into rectangles, \textcite{GGL1995packing} developed a set of heuristics, based on enumerating \emph{stable} solutions, where a circle is either on the ground or has two lower contacts. To get out of local optima, they apply a variety of approaches, like “shaking down” circles or apply genetic algorithms to improve the stable solutions.

\textcite{GLNO1998dense} propose a \emph{billiard simulation}, in which the circles are physically simulated as hard disks, and apply some additional steps to tighten the found packings.

\textcite{BDGL2000improving} model the disks as moving particles with repelling forces, and simulate them using a stragety resembling simmulated annealing}.

\textcite{WHZX2002improved} describe a “quasi-physical, quasi-human” approach, which first simulates the circles according to gravity and then finds circles which are "most squeezed", and re-insert them randomly.

\textcite{HLLX2006new} apply a quasi-human heuristic, which places new circles in gaps of approximately the same size as the circle, and combine it with a self look-ahead strategy which evalutates how beneficial the coice of a certain placement for each circle is.

\textcite{LB2013packing} apply a metaheuristic called \emph{formulation space search}, which alternates between two different heuristics, to avoid getting stuck in local minima. After that, they perform an improvement phase, which swaps circles.

\textcite{HHY2015action} start with a random configuration of unequal circles in a square, simulate them using a quasi-Newton method to reach a minimum energy state, and perform swapping steps based \emph{action spaces}, unoccupid rectangles, to get out of locally optimal solutions. Finally, they do a postprocessing step to increase the result's precision.

\paragraph{Pattern-based}

\textcite{GL1996repeated} indentify various regular patterns for packing equal circles in a square, which produce good results for instances where the number of circles equals certain forms like $k^2$, $k^2-1$, or $k(k+1)$.

\textcite{lubachevsky1997curved} studied \emph{curved hexagonal packings} of equal circles in a circle, a pattern derived from known best packings of circles in hexagons. For large $n$, a curved hexagonal packing was found to be not optimal. Instead, good packings found by the authors seemed to have a curved hexagonally-packed area in the center and an irregular pattern along the periphery.

%TODO \textcite{ZYC2015packing} extend the GVS to allow packing into \emph{damaged squares}, squares with some smaller squares removed from it. They enhance the GVS by Simulated Annealing and get significantly better results in their experiments compared to GVS.

\subsection{Global optimization methods}

Another much-explored idea is to model the packing problem as a set of (nonlinear) inequalities and solve the system using commercial solvers. 

\paragraph{Stochastic optimization}

\textcite{FMC2015evolutionary} solve the problem using Genetic Algorithms and other evolutionary approaches. As the evolved solutions are not always valid, they evalutate different repair strategies, like repulsion-based or Delauny-based methods.

\textcite{ZD2005effective} combine simulated annealing and tabu search to pack unequal circles into a circle.

\paragraph{Linear optimization}

\textcite{LIE2014approximate} compute approximated solutions to packing circles into rectangles by restricting their coordinates to a regular grid and formulate a binary LP, where the variables represent the assignment of the centers to the grid nodes.

\textcite{SY2004mathematical} formulate a method of transitioning from one local optimum to a better one when packing unequal circles into a strip, based on a reduced gradient method.

\paragraph{Nonlinear optimization}

\textcite{ALS2008disk} model the problem of packing circles into a circle as a global nonlinear optimization problem and apply a stochastic search method similar to Multistart, which they refer to as \emph{Monotonic Basin Hopping}, and which is inspired by molecular conformation problems. They also adapted this algorithm to pack circles into circular containers \cite{GJLS2009solving}.

\textcite{BS2008minimizing} design packing problems as a smooth nonlinear programming model, and solve them using continuous optimization. They give multiple formulations of the problem, allowing them to pack different-sized circles into circles, squares, strips, rectangles or equilateral triangles and to pack spheres into spheres, cuboids, tetrahedra, pyramids and cylinders.

\textcite{BG2010new} note that, given the contacts of the circle objects to the containers boundary beforehand, the numer of constraints is reduced significantly. They learn this information from previously best known solutions in the literature and solve the now overdetermined linear system to get high-precision results.

\subsection{Surveys}

\textcite{SMCSCG2007new} give an overview of numerous heuristics and optimization methods for packing equal circles in a square.

\textcite{HM2009literature} survey the literature on packing unequal circles into squares, rectangles, circles and other shapes, as well as on packing spheres into various shapes.
